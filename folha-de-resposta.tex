\documentclass[12pt,a4paper,twoside,twocolumn,table]{article}

%% isso ficou obsoleto -- desde 2020 o compilador usa utf8
% \usepackage[utf8]{inputenc}

%% Margens
%% NOTA: o uso das margens aqui (margens diferentes p/ frente e verso) está meio que assumindo que a primeira página é uma página da frente, e que o caderno seria grampeado na lateral inteira (tipo o do ENEM). Se só for grampeado na ponta, as margens seriam diferentes.
\usepackage[top=2cm, bottom=2cm, left=1.3cm, right=1.3cm]{geometry}
\usepackage[brazil]{babel}


%%% Cabeçalho e rodapé
\usepackage{fancyhdr}
\pagestyle{fancy}
\fancyhf{}

\renewcommand{\headrulewidth}{0pt}
\fancyfoot[C]{\includegraphics[height=8mm]{images/logo-simulado.png}}

%%% Varias colunas
\usepackage{multicol}

%%% Referência à última página
\usepackage{lastpage}

%%% Fonte sem serifa
\usepackage{lmodern}
\usepackage{sansmathfonts}
\renewcommand*{\familydefault}{\sfdefault}

%%% Imagens
\usepackage{graphicx}

%%% Questão
\usepackage{enumitem}
\usepackage{tikz}

\usepackage{import}

\begin{document}
 
\onecolumn

\newcommand*{\gcircled}[1]{%
  \tikz[baseline=(char.base)]{
    \node[
      shape=circle,
      draw,
      inner sep=0.5pt
    ] (char) {#1};
  }
}

\newcommand\thing{
  
  \setlength{\tabcolsep}{5pt}
  \begin{tabular}{m{2pt}|c}
    \hspace*{-2ex}\raggedleft
    {\footnotesize \the\numexpr \value{gabacol}*20 + \value{gabarow} + 1} &%
    \setlength{\tabcolsep}{0pt}%
    \begin{tabular}{ccccc}
      \gcircled{\scriptsize A} &
      \gcircled{\scriptsize B} &
      \gcircled{\scriptsize C} &
      \gcircled{\scriptsize D} & 
      \gcircled{\scriptsize E}%
    \end{tabular}%
    \hspace*{-2ex}
  \end{tabular}
  \stepcounter{gabacol}
}

\newcommand{\ghead}{\scriptsize \textbf{QUESTÃO/RESPOSTA}}
\newcounter{gabarow}
\newcounter{gabacol}[gabarow]

{
  \setlength{\tabcolsep}{0pt}
  \setlength\arrayrulewidth{2pt}
  \begin{tabular}{r}
    \huge CARTÃO-RESPOSTA
    \\ \hline
    \scriptsize SIMULADO DO EXAME NACIONAL DO ENSINO MÉDIO
  \end{tabular}
}

\bigskip
\begin{center}
  \noindent\textbf{\small Nome completo:}\\
  {
    \setlength{\tabcolsep}{0pt}
    \newlength{\nwid}
    \setlength{\nwid}{5mm}
    \renewcommand{\arraystretch}{1.2}
    \begin{tabular}{|m\nwid|m\nwid|m\nwid|m\nwid|m\nwid|m\nwid|m\nwid|m\nwid|m\nwid|m\nwid|m\nwid|m\nwid|m\nwid|m\nwid|m\nwid|m\nwid|m\nwid|m\nwid|m\nwid|m\nwid|m\nwid|m\nwid|}
      \hline &&&&&&&&&&&&&&&&&&&&& \\
      \hline &&&&&&&&&&&&&&&&&&&&& \\
      \hline
    \end{tabular}
  }
\end{center}

\bigskip
\begin{center}
  \large
  \textbf{INSTRUÇÕES}
\end{center}

\begin{multicols}{2}
  \footnotesize
  \raggedright
  \begin{enumerate}[noitemsep, topsep=1ex]
    \item Preencha o seu nome completo, com letras de forma, a sua data de nascimento e a sua cidade. Sua série a turma e o número (Se for aluno regularmente matriculado em alguma unidade do Sistema equipe de Ensino).
    \item Transcreva a frase apresentada na \textbf{CAPA DO SEU CADERNO DE QUESTÕES} no local abaixo indicado.
    \item Não haverá substituição do \textbf{CARTÃO-RESPOSTA}, por isso tenha muita atenção durante a marcação das alternativas.
    \item Em hipótese alguma, você devera deixar a sala de aplicação do exame portando o \textbf{CARTÃO-RESPOSTA}.
    \item O \textbf{CARTÃO-RESPOSTA/FOLHA DE REDAÇÃO} é o único documento que será utilizado para a correção de suas provas. Não o amasse, não o dobre, nem o rasure. O preenchimento deve ser feito com caneta esferográfica de tinta preta fabricada em material transparente. Não utilize caneta de outra cor, lápis ou lapiseira.
  \end{enumerate}
\end{multicols}

\arrayrulecolor{gray!40}
\begin{center}
  \setlength{\tabcolsep}{0pt}
  \setlength\arrayrulewidth{2pt}
  \begin{tabular}{|l|m{0.6\linewidth}|}
    \hline
    \colorbox{gray!15}{
    \parbox{0.35\linewidth}{
    \scriptsize\raggedright
    ATENÇÃO: TRANSCREVA AQUI COM A SUA CALIGRAFIA USUAL, A FRASE APRESENTADA NA CAPA DO SEU CADERNO DE QUESTÕES CONFORME AS INSTRUÇÕES NELA CONTIDAS.
    }
    } &  \\ \hline
  \end{tabular}
\end{center}

\arrayrulecolor{black}
\begin{center}
\noindent\colorbox{gray!15}{
    \begin{tabular}{m{0.57\linewidth}m{0.35\linewidth}}
      \small\raggedright
      Para todas as marcações neste \textbf{CARTÃO-RESPOSTA}, preencha os
      círculos completamente e com nitidez, utilizando \textbf{caneta esferográfica de tinta preta fabricada em material transparente}, conforme na ilustração.
     &
    \resizebox{0.35\textwidth}{!}{\import{images/}{preenchimento.pdf_tex}}
    \end{tabular}
}
\end{center}

\hrule

\rowcolors{2}{gray!25}{white}
\begin{center}
\begin{tabular}{|c||c||c||c||c|}\hline
  \rowcolor{gray!25}
  \ghead & \ghead & \ghead & \ghead & \ghead \\ \hline
  \thing & \thing & \thing & \thing & \thing \stepcounter{gabarow}\\ \hline
  \thing & \thing & \thing & \thing & \thing \stepcounter{gabarow}\\ \hline
  \thing & \thing & \thing & \thing & \thing \stepcounter{gabarow}\\ \hline
  \thing & \thing & \thing & \thing & \thing \stepcounter{gabarow}\\ \hline
  \thing & \thing & \thing & \thing & \thing \stepcounter{gabarow}\\ \hline
  \thing & \thing & \thing & \thing & \thing \stepcounter{gabarow}\\ \hline
  \thing & \thing & \thing & \thing & \thing \stepcounter{gabarow}\\ \hline
  \thing & \thing & \thing & \thing & \thing \stepcounter{gabarow}\\ \hline
  \thing & \thing & \thing & \thing & \thing \stepcounter{gabarow}\\ \hline
  \thing & \thing & \thing & \thing & \thing \stepcounter{gabarow}\\ \hline
  \thing & \thing & \thing & \thing & \thing \stepcounter{gabarow}\\ \hline
  \thing & \thing & \thing & \thing & \thing \stepcounter{gabarow}\\ \hline
  \thing & \thing & \thing & \thing & \thing \stepcounter{gabarow}\\ \hline
  \thing & \thing & \thing & \thing & \thing \stepcounter{gabarow}\\ \hline
  \thing & \thing & \thing & \thing & \thing \stepcounter{gabarow}\\ \hline
  \thing & \thing & \thing & \thing & \thing \stepcounter{gabarow}\\ \hline
  \thing & \thing & \thing & \thing & \thing \stepcounter{gabarow}\\ \hline
  \thing & \thing & \thing & \thing & \thing \stepcounter{gabarow}\\ \hline
  \thing & \thing & \thing & \thing & \thing \stepcounter{gabarow}\\ \hline
  \thing & \thing & \thing & \thing & \thing \\ \hline
\end{tabular}
\end{center}

\end{document}
