\documentclass[12pt,a4paper,twoside,twocolumn]{article}

%% isso ficou obsoleto -- desde 2020 o compilador usa utf8
% \usepackage[utf8]{inputenc}

%% Margens
%% NOTA: o uso das margens aqui (margens diferentes p/ frente e verso) está meio que assumindo que a primeira página é uma página da frente, e que o caderno seria grampeado na lateral inteira (tipo o do ENEM). Se só for grampeado na ponta, as margens seriam diferentes.
\usepackage[top=2cm, bottom=2cm, left=1.5cm, right=1.2cm]{geometry}

%% Separador entre colunas
\setlength{\columnseprule}{0.4pt}
\setlength{\columnsep}{6mm}

%%% Referência à última página
\usepackage{lastpage}

%%% Fonte sem serifa
\usepackage{lmodern}
\renewcommand*{\familydefault}{\sfdefault}

%%% Imagens
\usepackage{graphicx}

%%% Cabeçalho e rodapé
\usepackage{fancyhdr}
\pagestyle{fancy}
\fancyhf{}

% Cabeçalho
\fancyhead[C]{\includegraphics[height=8mm]{images/logo-simulado.png}}

% Rodapé
\fancyfoot[LE,RO]{\thepage~de~\pageref{LastPage}}
\fancyfoot[C]{\small Simulado Emancipa -- Caderno único}
\fancyfoot[RE,LO]{\small \includegraphics[height=12pt]{images/logo-simulado.png}}

%%% Questão
\usepackage{enumitem}
\usepackage{tikz}

\newcounter{questao}
\setcounter{questao}{1} 
\newcommand{\questao}{
  \noindent\textbf{QUESTÃO \arabic{questao}}
  \par
  \vspace{-8pt}
  \noindent\rule[3pt]{\columnwidth}{1pt}
  \stepcounter{questao}
}

% Circula alguma letra
\newcommand*{\circled}[1]{%
  \tikz[baseline=(char.base)]{
    \node[
      shape=circle,
      fill,
      text=white,
      font=\bfseries,
      inner sep=0.5pt
    ] (char) {#1};
  }
}

% Alternativas
\newenvironment{alternativas}{%
  \begin{enumerate}[
    label=\protect\circled{\scriptsize\Alph*},
    leftmargin=8mm,
    itemsep=-3pt,
    topsep=1ex
    ]
  }{%
  \end{enumerate}
}

\usepackage{lipsum} % for some dummy text

\begin{document}

\questao
\lipsum[4]
\begin{alternativas}
  \item \lipsum[1][1]
  \item \lipsum[1][2]
  \item \lipsum[1][3]
  \item \lipsum[1][4]
\end{alternativas}

\questao
\lipsum[8]
\begin{alternativas}
  \item \lipsum[9][1]
  \item \lipsum[9][2]
  \item \lipsum[9][3]
  \item \lipsum[9][4]
\end{alternativas}

\questao
\lipsum[4]
\begin{alternativas}
  \item \lipsum[2][1]
  \item \lipsum[3][2]
  \item \lipsum[8][3]
  \item \lipsum[3][4]
\end{alternativas}

\questao
\lipsum[4]
\begin{alternativas}
  \item \lipsum[1][1]
  \item \lipsum[1][2]
  \item \lipsum[1][3]
  \item \lipsum[1][4]
\end{alternativas}

\questao
\lipsum[8]
\begin{alternativas}
  \item \lipsum[9][1]
  \item \lipsum[9][2]
  \item \lipsum[9][3]
  \item \lipsum[9][4]
\end{alternativas}

\questao
\lipsum[4]
\begin{alternativas}
  \item \lipsum[2][1]
  \item \lipsum[3][2]
  \item \lipsum[8][3]
  \item \lipsum[3][4]
\end{alternativas}
\end{document}
