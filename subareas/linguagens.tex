\noindent{\small\textbf{LINGUAGENS, CÓDIGOS E SUAS TECNOLOGIAS}}

\noindent\textbf{Questões 41 a 60} %arrumar o número

% (ENEM - 2018 - PROVA AMARELA)
\citacao{
\begin{verse}
Don't write in English, they said, \\
English is not your mother tongue\dots \\
\dots The language I speak \\
Becomes mine, its distortions, its queerness \\
All mine, mine alone, it is half English, half \\
Indian, funny perhaps, but it is honest, \\
It is as human as I am human\dots \\
\dots It voices my joys, my longings my \\
Hopes\dots

(Kamala Das, 1965:10)
\end{verse}
}{GARGESH, R. South Asian Englishes. In: KACHRU, Y; NELSON, C. L. (Eds.). The Handbook of World English. Singapore: Blackwell, 2006}
A poetisa Kamala Das, como muitos escritores indianos, escreve suas obras em inglês, apesar de essa não ser sua primeira língua. Nesses versos, ela
\begin{alternativas}
\item usa a língua inglesa como efeito humorístico.
\item recorre a vozes de vários escritores ingleses.
\item adverte sobre o uso distorcido da língua inglesa.
\item demonstra consciência de sua identidade linguística.
\item reconhece a incompreensão na sua maneira de falar inglês.
\end{alternativas}

(ENEM - 2019 - PROVA AMARELA)
Is this life
Sitting on a park bench
Thinking about a friend of mine
He was only twenty-three
Gone before he had his time
It came without a warning
Didn’t want his friend to see him cry
he knew the day was dawning
And I didn’t have a chance to say goodbye

MADONNA. Erotica. Estados Unidos. Marverick, 1992.

A canção, muitas vezes, é uma forma de manifestar sentimentos e emoções da vida cotidiana. Por exemplo, o sofrimento retratado nessa canção foi causado:

A - pela morte precoce de um amigo jovem.
B - pelo término de um relacionamento amoroso.
C - pela mudança de um amigo para outro país.
D - pelo fim de uma amizade de mais de vinte anos.
E - pela traição por parte de pessoa próxima.
